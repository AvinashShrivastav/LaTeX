\documentclass{book}

\usepackage{subcaption}
\usepackage{graphicx}

\begin{document}
        
\begin{figure}
\centering
\begin{subfigure}[b]{0.49\textwidth}
    \includegraphics[width=\linewidth,height=5cm]{figure.png}
    \caption{Subfigure 1 caption}
    \label{fig:first}
\end{subfigure}
\hfill
\begin{subfigure}[b]{0.49\textwidth}
    \includegraphics[width=\linewidth,height=5cm]{figure.png}
    \caption{Subfigure 2 caption}
    \label{fig:second}
\end{subfigure}

\medskip 

\begin{subfigure}[b]{0.49\textwidth}
    \includegraphics[width=\linewidth,height=5cm]{figure.png}
    \caption{Subfigure 3 caption}
    \label{fig:third}
\end{subfigure}
\hfill
\begin{subfigure}[b]{0.49\textwidth}
    \includegraphics[width=\linewidth,height=5cm]{figure.png}
    \caption{Subfigure 4 caption}
    \label{fig:fourth} 
\end{subfigure}
        
\caption{This is a figure containing several subfigures.}
\label{fig:figures} 

\vspace{12pt}
In the text, you can refer to subfigures of figure 1 as \ref{fig:first}, \ref{fig:second}, \ref{fig:third}, and \ref{fig:fourth}\par
and to the sub-index as (a), (b), (c), and (d).

\end{figure}

\end{document}